\section{Python Basics}\label{python-basics}

\subsection{Control Structures}\label{control-structures}

\subsubsection{The for loop}\label{the-for-loop}

    We have already introduced the \texttt{while} loop, which is an example
of a control structure. For this module, we will only need two other
control structures, namely the \texttt{for} loop and the \texttt{if}
\texttt{else} structure. Let us look at a simple example of a
\texttt{for} loop:

    \begin{tcolorbox}[breakable, size=fbox, boxrule=1pt, pad at break*=1mm,colback=cellbackground, colframe=cellborder]
\prompt{In}{incolor}{1}{\boxspacing}
\begin{Verbatim}[commandchars=\\\{\}]
\PY{k}{for} \PY{n}{i} \PY{o+ow}{in} \PY{n+nb}{range}\PY{p}{(}\PY{l+m+mi}{10}\PY{p}{)}\PY{p}{:}
    \PY{n}{cube} \PY{o}{=} \PY{n}{i}\PY{o}{*}\PY{o}{*}\PY{l+m+mi}{3}
    \PY{n+nb}{print}\PY{p}{(}\PY{l+s+s2}{\PYZdq{}}\PY{l+s+s2}{The cube of }\PY{l+s+s2}{\PYZdq{}} \PY{o}{+} \PY{n+nb}{str}\PY{p}{(}\PY{n}{i}\PY{p}{)} \PY{o}{+} \PY{l+s+s2}{\PYZdq{}}\PY{l+s+s2}{ is }\PY{l+s+s2}{\PYZdq{}} \PY{o}{+} \PY{n+nb}{str}\PY{p}{(}\PY{n}{cube}\PY{p}{)}\PY{p}{)}
\end{Verbatim}
\end{tcolorbox}

    Note again how we used indentation to denote the scope of the
\texttt{for} loop. Similar to the while loop, the indentation started
after the colon (":") symbol. Note also that in this case, it is not
necessary to write \texttt{list(range(10))}. Instead of
\texttt{range(10)} you could of course also provide any list of numbers
you are interested in knowing the cube of:

    \begin{tcolorbox}[breakable, size=fbox, boxrule=1pt, pad at break*=1mm,colback=cellbackground, colframe=cellborder]
\prompt{In}{incolor}{2}{\boxspacing}
\begin{Verbatim}[commandchars=\\\{\}]
\PY{k}{for} \PY{n}{i} \PY{o+ow}{in} \PY{p}{[}\PY{l+m+mf}{2.3}\PY{p}{,} \PY{l+m+mf}{6.23}\PY{p}{,} \PY{o}{\PYZhy{}}\PY{l+m+mi}{3}\PY{p}{,} \PY{l+m+mi}{19}\PY{p}{,} \PY{l+m+mi}{0}\PY{p}{,} \PY{o}{\PYZhy{}}\PY{l+m+mi}{12}\PY{p}{]}\PY{p}{:}
    \PY{n}{cube} \PY{o}{=} \PY{n}{i}\PY{o}{*}\PY{o}{*}\PY{l+m+mi}{3}
    \PY{n+nb}{print}\PY{p}{(}\PY{l+s+s2}{\PYZdq{}}\PY{l+s+s2}{The cube of }\PY{l+s+s2}{\PYZdq{}} \PY{o}{+} \PY{n+nb}{str}\PY{p}{(}\PY{n}{i}\PY{p}{)} \PY{o}{+} \PY{l+s+s2}{\PYZdq{}}\PY{l+s+s2}{ is }\PY{l+s+s2}{\PYZdq{}} \PY{o}{+} \PY{n+nb}{str}\PY{p}{(}\PY{n}{cube}\PY{p}{)}\PY{p}{)}
\end{Verbatim}
\end{tcolorbox}

    \begin{Verbatim}[commandchars=\\\{\}]
The cube of 2.3 is 12.166999999999998
The cube of 6.23 is 241.80436700000004
The cube of -3 is -27
The cube of 19 is 6859
The cube of 0 is 0
The cube of -12 is -1728
    \end{Verbatim}

    Equivalently, if you already have a list of number, you may write

    \begin{tcolorbox}[breakable, size=fbox, boxrule=1pt, pad at break*=1mm,colback=cellbackground, colframe=cellborder]
\prompt{In}{incolor}{3}{\boxspacing}
\begin{Verbatim}[commandchars=\\\{\}]
\PY{n}{mylist}\PY{o}{=}\PY{p}{[}\PY{l+m+mf}{2.3}\PY{p}{,} \PY{l+m+mf}{6.23}\PY{p}{,} \PY{o}{\PYZhy{}}\PY{l+m+mi}{3}\PY{p}{,} \PY{l+m+mi}{19}\PY{p}{,} \PY{l+m+mi}{0}\PY{p}{,} \PY{o}{\PYZhy{}}\PY{l+m+mi}{12}\PY{p}{]}
\PY{k}{for} \PY{n}{i} \PY{o+ow}{in} \PY{n}{mylist}\PY{p}{:}
    \PY{n}{cube} \PY{o}{=} \PY{n}{i}\PY{o}{*}\PY{o}{*}\PY{l+m+mi}{3}
    \PY{n+nb}{print}\PY{p}{(}\PY{l+s+s2}{\PYZdq{}}\PY{l+s+s2}{The cube of }\PY{l+s+s2}{\PYZdq{}} \PY{o}{+} \PY{n+nb}{str}\PY{p}{(}\PY{n}{i}\PY{p}{)} \PY{o}{+} \PY{l+s+s2}{\PYZdq{}}\PY{l+s+s2}{ is }\PY{l+s+s2}{\PYZdq{}} \PY{o}{+} \PY{n+nb}{str}\PY{p}{(}\PY{n}{cube}\PY{p}{)}\PY{p}{)}
\end{Verbatim}
\end{tcolorbox}

    \begin{Verbatim}[commandchars=\\\{\}]
The cube of 2.3 is 12.166999999999998
The cube of 6.23 is 241.80436700000004
The cube of -3 is -27
The cube of 19 is 6859
The cube of 0 is 0
The cube of -12 is -1728
    \end{Verbatim}

